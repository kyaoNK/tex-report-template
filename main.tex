\documentclass[a4j,uplatex,dvipdfmx,11pt]{jsarticle}    % 文章クラスとオプション設定
%%%%%%%%%% スタイル %%%%%%%%%%%%%%%%%%%%%%%%%%%%%%%%%%%%%%%%%%%%%%%%%%%%%%%%%%%%%%%%%%%%%%%%%%%%%%

% 複数行コメント
\usepackage{comment}    

% 余白の調整
\usepackage[top=30truemm,bottom=30truemm,left=20truemm,right=20truemm]{geometry}

% 数式関連
\usepackage{amsmath,amssymb,nccmath}    % 数式
\usepackage{bm}         % ベクトルの太文字
\newcommand{\alignref}[1]{\textbf{式(\ref{#1})}}    % 数式参照

% 画像関連
\usepackage{graphicx} % 画像
\usepackage{here}% figureの位置調整
\renewcommand{\figurename}{\textbf{Fig.}}    % 画像キャプション
\newcommand{\figurenameref}[1]{\textbf{Fig.\ref{#1}}}   % 画像の参照を定義

% 表設定
\usepackage{multirow}   % 表の行の結合
\usepackage{longtable}  % ページをまたぐ長い表
\renewcommand{\tablename}{\textbf{Table.}}   % 表キャプション
\newcommand{\tablenameref}[1]{\textbf{Table.\ref{#1}}}  % 表の参照を定義

% ソースコード
\usepackage{listings,plistings}
\renewcommand{\lstlistingname}{\textbf{Source Code.}}    % ソースコードキャプション
\newcommand{\lstlistingnameref}[1]{\textbf{Source Code.\ref{#1}}}   % ソースコード参照

\lstset{
    language=C++,       % 言語設定
    frame=shadowbox,    % 枠設定
    breaklines=true,    % 行が長くなった場合自動改行
    breakindent=12pt,   % 自動改行時のインデント
    columns=fixed,      % 文字の間隔を統一
    basewidth=0.5em,    % 文字の横のサイズを小さく
    numbers=left,       % 行数の位置
    numberstyle={\scriptsize},  % 行数のフォント
    stepnumber=1,       % 行数の増間
    numbersep=1zw,      % 行数の余白
    xrightmargin=0zw,   % 左の余白
    xleftmargin=2zw,    % 右の余白
    framexleftmargin=18pt,  
    keepspaces=true,    % スペースを省略せず保持
    lineskip=-0.2ex,    % 枠線の途切れ防止
    tabsize = 4,        % タブ数
    showstringspaces=false,  %文字列中の半角スペースを表示させない
    basicstyle     ={\small\ttfamily},             % 基礎の文字のフォント設定
    identifierstyle={\small},           % 変数名などのフォント設定
    keywordstyle={\small\bfseries},     % 予約語などのフォント設定
}
%%%%%%%%%% 本文 %%%%%%%%%%%%%%%%%%%%%%%%%%%%%%%%%%%%%%%%%%%%%%%%%%%%%%%%%%%%%%%%%%%%%%%%%%%%%%%%%

\begin{document}

% !TEX root = main.tex

%%%%%%%%%%%%%%%%%%%%%%%%%%%%%%%%%%%%%%%%%%%%%%%%%%%%%%%%%%%%%%%%%%%%%%%%%%%%%%%%%%%%%%%%%%%%%%%%

\title{タイトル}
\author{氏名\thanks{所属}}
\date{年月日}
\maketitle

%%%%%%%%%%%%%%%%%%%%%%%%%%%%%%%%%%%%%%%%%%%%%%%%%%%%%%%%%%%%%%%%%%%%%%%%%%%%%%%%%%%%%%%%%%%%%%%%

\tableofcontents
\newpage

% !TEX root = main.tex

%%%%%%%%%%%%%%%%%%%%%%%%%%%%%%%%%%%%%%%%%%%%%%%%%%%%%%%%%%%%%%%%%%%%%%%%%%%%%%%%%%%%%%%%%%%%%%%%
\section{実験目的}
%%%%%%%%%%%%%%%%%%%%%%%%%%%%%%%%%%%%%%%%%%%%%%%%%%%%%%%%%%%%%%%%%%%%%%%%%%%%%%%%%%%%%%%%%%%%%%%%

% !TEX root = main.tex

%%%%%%%%%%%%%%%%%%%%%%%%%%%%%%%%%%%%%%%%%%%%%%%%%%%%%%%%%%%%%%%%%%%%%%%%%%%%%%%%%%%%%%%%%%%%%%%%
\section{理論}
%%%%%%%%%%%%%%%%%%%%%%%%%%%%%%%%%%%%%%%%%%%%%%%%%%%%%%%%%%%%%%%%%%%%%%%%%%%%%%%%%%%%%%%%%%%%%%%%

% !TEX root = main.tex

%%%%%%%%%%%%%%%%%%%%%%%%%%%%%%%%%%%%%%%%%%%%%%%%%%%%%%%%%%%%%%%%%%%%%%%%%%%%%%%%%%%%%%%%%%%%%%%%
\section{実験環境}
%%%%%%%%%%%%%%%%%%%%%%%%%%%%%%%%%%%%%%%%%%%%%%%%%%%%%%%%%%%%%%%%%%%%%%%%%%%%%%%%%%%%%%%%%%%%%%%%


% !TEX root = main.tex

%%%%%%%%%%%%%%%%%%%%%%%%%%%%%%%%%%%%%%%%%%%%%%%%%%%%%%%%%%%%%%%%%%%%%%%%%%%%%%%%%%%%%%%%%%%%%%%%
\section{実験課題}
%%%%%%%%%%%%%%%%%%%%%%%%%%%%%%%%%%%%%%%%%%%%%%%%%%%%%%%%%%%%%%%%%%%%%%%%%%%%%%%%%%%%%%%%%%%%%%%%


% !TEX root = main.tex

%%%%%%%%%%%%%%%%%%%%%%%%%%%%%%%%%%%%%%%%%%%%%%%%%%%%%%%%%%%%%%%%%%%%%%%%%%%%%%%%%%%%%%%%%%%%%%%%
\section{実験方法}
%%%%%%%%%%%%%%%%%%%%%%%%%%%%%%%%%%%%%%%%%%%%%%%%%%%%%%%%%%%%%%%%%%%%%%%%%%%%%%%%%%%%%%%%%%%%%%%%


% !TEX root = main.tex

%%%%%%%%%%%%%%%%%%%%%%%%%%%%%%%%%%%%%%%%%%%%%%%%%%%%%%%%%%%%%%%%%%%%%%%%%%%%%%%%%%%%%%%%%%%%%%%%
\section{実験結果}
%%%%%%%%%%%%%%%%%%%%%%%%%%%%%%%%%%%%%%%%%%%%%%%%%%%%%%%%%%%%%%%%%%%%%%%%%%%%%%%%%%%%%%%%%%%%%%%%

% !TEX root = main.tex

%%%%%%%%%%%%%%%%%%%%%%%%%%%%%%%%%%%%%%%%%%%%%%%%%%%%%%%%%%%%%%%%%%%%%%%%%%%%%%%%%%%%%%%%%%%%%%%%
\section{考察}
%%%%%%%%%%%%%%%%%%%%%%%%%%%%%%%%%%%%%%%%%%%%%%%%%%%%%%%%%%%%%%%%%%%%%%%%%%%%%%%%%%%%%%%%%%%%%%%%


%%%%%%%%%% 参考文献 %%%%%%%%%%%%%%%%%%%%%%%%%%%%%%%%%%%%%%%%%%%%%%%%%%%%%%%%%%%%%%%%%%%%%%%%%%%%%%

% !TEX root = main.tex

%%%%%%%%%% 参考文献 %%%%%%%%%%%%%%%%%%%%%%%%%%%%%%%%%%%%%%%%%%%%%%%%%%%
\begin{thebibliography}{99}
    \bibitem{ラベル}資料名
\end{thebibliography}


\end{document}